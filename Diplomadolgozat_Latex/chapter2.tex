%----------------------------------------------------------------------------
\chapter{Célkitűzések} \label{chapter2}
%----------------------------------------------------------------------------

A dolgozat célja a modern technológiai elvárásoknak megfelelő jelenlétkezelő rendszer megvalósítása, amely kielégíti rohamosan fejlődő digitális világunk elvárásait mint megvalósítás, mint pedig használhatóság szempontjából. 
Egy olyan rendszer kidolgozása a cél, amely növeli az órai jelenlétkezelés hatékonyságát. Ez elsősorban időhatékonyságot és gyors működést feltételez, mivel az a cél, hogy az óra időtartamából a lehető legkevesebbet tegye ki a jelenlétek menedzselése.

A rendszer megvalósítása során külön figyelmet kell szentelnünk annak, hogy használata olyan erőforrásokat igényeljen, ami mindenki számára elérhető, mellőzve a jelentős anyagi vonzatát. Mindezt figyelembe véve jutottunk arra a következtetésre, hogy az okostelefon jelenti a leghatékonyabb eszközt a rendszer használatára, mivel elmondható, hogy már minden diák rendelkezik vele. A fő modul, ami a diákok detektálását és azonosítását végzi legyen minél kevesebb beavatkozást igénylő megoldás, ami gyors és hatékony is egyben.

Mindezek mellett fontos cél az is, hogy a jelenléteket ne csak rögzítsük, hanem elérhetővé is tegyük a tanárok és diákok számára, hogy azok bármikor megtekinthetőek legyenek. A megtekinthetőség mellett az is nagyon fontos, hogy a tanárnak lehetősége legyen lementeni a jelenléteket, tantárgyakra lebontva, választható formátumban.

A rendszer kiegészítő funkcionalitásaként, az online oktatási folyamatokat egyszerűsítő eszközöket is szeretnénk elérhetővé tenni az alkalmazáson belül. Ez azt jelenti, hogy minden tantárgyhoz, ami megjelenik a rendszerben, rendelünk egy Google tantermet (Google Classroom), így azonnal elérhetővé válik, amennyiben a tanár használni szeretné. Az e-mail küldés, mint kiegészítő funkciónak is hasznos szerepe van az online világ hatékony működésében, így ezen folyamat leegyszerűsítése végett implementálunk egy olyan opciót, ami lehetővé teszi a tanár és diák számára az e-mail küldést. Abban különbözik a sima e-mail küldési lehetőségtől, hogy a diáknak nem kell tudnia a tanár e-mail címét, kiválasztva a nevét máris tud üzenetet küldeni. A tanár esetében ez a plusz azt jelenti, hogy egyszerűen tud csoportos e-mailt küldeni méghozzá a szak és tantárgy kiválasztásával.

Amennyiben a célkitűzéseinket sikerül megvalósítani, már egy olyan jelenlétkezelő rendszert tudunk biztosítani, ami megállja a helyét az egyetemi szférában is, kiemelve a hatékonyságot, időspórlást és a gyors működést. Emellett a továbbfejlesztési lehetőségek tárháza is végtelen, amelyet a jövőben az igények és visszajelzések alapján szeretnénk megvalósítani.



