%----------------------------------------------------------------------------
\chapter{Összefoglaló} \label{chapter9}
%----------------------------------------------------------------------------

A dolgozat során egy olyan szoftverrendszer kivitelezésére került sor, amely megpróbál eleget tenni a modernkori jelenlétkezelés elvárásainak, biztosítva a hatékony és megbízható működést, használata pedig az erre szánt időt minimálisra csökkenti, de ugyanakkor felhasználóbarát megjelenést kölcsönöz, mint a diák, mint pedig a tanárok számára.

Szoftverünk három fő modul köré épül. A diákok képeinek mentése és a jelenlétek bevitele az arcfelismerő és életszerűség-érzékelő modul részeit képezik. Ezáltal lehetőség van a fényképek egyszerű mentésére, valamint ami a legfontosabb, a jelenlétek gyors és biztonságos bevitelére. Emellett a Diák applikáció lehetővé teszi a hallgatók számára a jelenlétek utánkövetését, ugyanakkor e-mail küldését a tanárnak, valamint a Neptun és Google tanterembe való belépést is. A Tanár alkalmazásban is van lehetőség a korábbi jelenlétek megtekintésére, ezek kiexportálására több formátumban, illetve ugyanúgy vannak a kapcsolattartást elősegítő megoldások, mint például az e-mail küldés a diákoknak, naptáresemény létrehozása és kiküldése, valamint a Google tanterembe való belépés is adott. 

Amint azt a tapasztalataink is mutatják, fontos, hogy a folyamatok leegyszerűsítéséhez és felgyorsításához bátran használjuk az újszerű technológiákat, és haladjunk a modernizálás irányába. Ezáltal hatékonyabban és gyorsabban tudjuk elvégezni a repetitív munkákat és több idő marad azokra a dolgokra, amelyek a fejlődésünket szolgálják.

Összességében elmondhatjuk, hogy sikerült kivitelezni egy olyan élő arcfelismerés-alapú jelenlétkezelő alkalmazást, amely a robusztus működés mellett eleget tesz a 21. század technológiai elvárásainak, mindazonáltal egyszerűen kezelhető, mindenki számára elérhető, valamint nem igényel különleges szoftver és hardver készletet sem.
