%----------------------------------------------------------------------------
\chapter{Konferencia tapasztalatok} \label{chapter4}
%----------------------------------------------------------------------------

Amint azt korábban említettem, dolgozatom alaptémáját a korábbi licenszdolgozatom adta. A \textit{Jelenlétkezelő alkalmazás} című projektemmel részt vettünk a 
\href{https://mtdk.tmd.ro/index.php/site/page/?p=21}{XXII. Műszaki Tudományos Diákköri Konferencián}, alkalmazott informatika tagozaton, ahol II. helyezést értünk el. Emellett sikerült bemutatnunk a dolgozatot a \href{https://ojs.emt.ro/enelko-szamokt/article/view/327/266}{XXI. Energetika-Elektrotechnika – ENELKO és XXX. Számítástechnika és Oktatás – SzámOkt Multi-konferencia} keretein belül, ahol a dolgozat alapján megírt cikk is publikálásra került.

A tudományos bizottság mindkét esetben a biztonságot, és ami még fontosabb a rendszer átjátszhatóságát boncolgatta. Ahhoz, hogy a rendszer használható legyen, illetve ne sértsen semmilyen személyi jogokat, az átjátszhatóságot feláldoztuk ennek az oltárán. Az említett rendszer nem igényelt semmiféle olyan adatot, ami GDPR problémába ütközne, működéséhez csupán egy okostelefonra volt szükség, és bár bizonyos mértékben támadható volt, tettünk némi lépést afelé, hogy a kijátszhatóság lehetőségét a minimálisra csökkentsük. Ilyen intézkedés volt például, hogy a diák azonosítására szolgáló adatok QR kód formájában kerülnek titkosításra, amelyből csak a megfelelő alkalmazással és adatfeldolgozással kerülnek feldolgozásra. Emellett a regisztrációkor elmentettük a diák által használt eszköz egyedi azonosítóját, ami ugyanúgy kódolva lett a diákot azonosító QR kódban, ezáltal minden jelentkezéskor ellenőrizhető volt, hogy ugyanarról a telefonról lépik be adott azonosítójú diák. A nagyobb biztonság elérése érdekében a Neptun azonosítókat is ellenőriztük, ezáltal tudtuk azt kiszűrni, hogy egy alkalmazáson belül csak ugyanazzal az azonosítóval rendelkező diák tudja jelezni ottlétét.

Elmondhatjuk, hogy bár megpróbáltunk minden lehetséges támadást kivédeni, még mindig vannak olyan lehetőségek, amik teret adnak a kijátszhatóságnak, ezáltal a rendszerünk kevésbé lesz megbízható. Emellett az is fontos szerepet játszik a rendszer hatékonyságát illetően, hogy a jelenlétek bevitele szükségessé teszi a tanár és diák interakcióját is, ezáltal bár a hagyományos módszernél gyorsabb, mégsem bizonyul a leghatékonyabb eljárásnak.

A kijátszhatóság és biztonság fogalmakat tekintve a bizottság szembesített minket néhány olyan kérdéssel, amelyre a jelenlegi rendszerünkkel szeretnénk választ adni, kiküszöbölve ezzel az esetleges támadási lehetőségeket.

\newpage

Néhány nagyobb kérdés köré építettük fel a megfogalmazott problémákat:

\begin{enumerate}[label=(\alph*)]
    \item Hogyan tudom biztosítani, hogy egy diák csak egyszer tudjon jelentkezni saját alkalmazásán belül?
    \item Hogyan tudom kiküszöbölni azt, hogy az azonosító továbbküldésével egy diák mások helyett is jelentkezni tudjon?
    \item Hogyan érhetjük el azt, hogy a hatékonyságot növelve a tanár/diák beavatkozása nélkül is megbízhatóan működjön a rendszer?
    \item Gyorsaság szempontjából hogyan lehetne javítani a rendszer teljesítményén?
\end{enumerate}

Előrevetítve a dolgozatom keretein belül megvalósított rendszer megoldásait, a fent leírt kérdésekre az alábbi megoldásokkal tudunk válaszolni:

\begin{enumerate}[label=(\alph*)]
    \item Az arcfelismerés technikáját alkalmazva egy diák csak saját maga jelenlétét tudja igazolni.
    \item Az életszerűség-érzékelést alkalmazva, amennyiben egy fotót vagy videót használnak, a rendszer kiszűri a \enquote{hamis} arcokat.
    \item A hatékonyság növelése érdekében az interakciókat minimálisra csökkentettük, a diák részéről ez megszűnt, míg a tanárnak csupán egyszer kell beállítania az óra adatait, onnantól a rendszer robusztusan, külső beavatkozás nélkül funkcionál.
    \item Az interakciók kiiktatásával a jelenlétek bevitele sokkal gyorsabb lett, ugyanakkor egyszerre akár több diákot is tudunk azonosítani, ezzel is minimálisra csökkentve az erre szánt időt.
\end{enumerate}

A jelenlegi projekt keretein belül igyekeztünk egy olyan rendszert kialakítani, ami eleget tesz a megfogalmazott elvárásoknak, megbízható, robusztus és mindazonáltal minimális beavatkozást igényel. 