\pagenumbering{gobble}

\selectlanguage{magyar}
\hungarianParagraph

%----------------------------------------------------------------------------
% Abstract in Hungarian
%----------------------------------------------------------------------------

\chapter*{Kivonat}

Napjainkban az egyetemek többsége még mindig a hagyományos  módszereket használja a tanórák során a hallgatók jelenléteinek menedzselésére. Ezen módszerek legnagyobb hátránya, hogy kicsit sem hatékonyak, illetve utólag a jelenléti ívek legtöbb esetben nem érhetőek el a diákok számára. Ugyancsak hátrányként említhető az is, hogy a később érkező diákok megzavarják az órát a procedúra elvégzése során.

A projekt célja egy korábban megvalósított jelenlétkezelő szoftverrendszer hibáinak kiküszöbölése, ugyanakkor az azonosító modul lecserélése, aminek következtében rendszerünk megbízhatóbbá, gyorsabbá és robusztusabbá válik. 

A kidolgozott rendszer sajátossága, hogy a diák részéről nem szükséges semmiféle beavatkozás a jelenléte igazolására, csupán egy arcfelismerő kamera előtt kell elsétáljon. Ennek következtében az arcfelismerő algoritmus segítségével azonosítva lesz, valamint egy életszerűség-érzékelő algoritmus biztosítja a rendszer megbízhatóságát. Ugyanakkor lehetősége van a fejlesztett mobilapplikáció letöltésére, ahol saját jelenléteit tudja követni.

Ennek tekintetében a rendszer három alappillérből áll: a jelenlétkezelő modul, melynek alkalmazásával a diák azonosítva lesz, jelenléte pedig bekerül az adatbázisba. Mindemellett a korábbi rendszer két eleme a Tanár és Diák alkalmazás kisebb változtatásokkal ugyan de továbbra is betölti szerepét. Ezáltal mind a tanárnak, mind pedig a diáknak lehetősége van a jelenlétek követésére, illetve a tanár kezében a lehetőség, hogy mindezeket kimentse.

Az applikációk fontos szerepet játszanak a tanár-diák kapcsolattartás megkönnyítésében, így a jelenlétek utánkövetésén kívül továbbra is a választható opciók között szerepel az e-mail küldés, naptáresemény létrehozása, illetve a Google tanterembe való belépés lehetősége is.

A diákok azonosítására alkalmazott modul futtatásához egy számítógépre, valamint az applikáció használatához Android operációs rendszerrel rendelkező telefonokra van szükség. A használt adatbázis felhő alapú, amely könnyebb elérést és átjárhatóságot biztosít a különböző modulok között.


\vspace*{2cm}

\noindent \textbf{Kulcsszavak:} jelenlét, azonosítás, arcfelismerés, életszerűség-érzékelés, Android alkalmazás
\vfill
\selectlanguage{romanian}

%----------------------------------------------------------------------------
% Abstract in Romanian
%----------------------------------------------------------------------------
\chapter*{Rezumat}

În zilele noastre majoritatea universităților urmărește metoda veche în decursul orelor pentru a gestiona prezența studenților. Dezavantajul acestor metode este că nu sunt deloc eficiente, respectiv foaia de prezență nu rămâne accesibil pentru studenți. De asemenea faptul că studenții care întârzie, deranjează ceilalți colegi cu această procedură poate fi menționată iarăși ca un dezavantaj.

Scopul proiectului este de a elimina erorile dintr-un sistem software anterior dezvoltat, în același timp înlocuirea modulului de identificare, ceea ce face sistemul nostru mai fiabil, mai rapid și eficient. 

Particularitatea sistemului dezvoltat este că studenții nu au nevoie de nici o intervenție pentru a verificarea prezenței, trebuie doar să treacă în fața unei camere de recunoaștere facială. Ca urmare, acesta va fi identificat cu ajutorul algoritmului de recunoaștere facială, iar algoritmul de detectare a vitalității asigură fiabilitatea sistemului. In acelasi timp aveți opțiunea de a descărca aplicația mobilă dezvoltată, de unde se va putea urmări prezența.

În acest sens, sistemul este format din trei piloni de bază: modulul de management al prezenței, datorită cărui studentul va fi identificat, iar prezența acestuia va fi introdusă în baza de date. In plus
cele două elemente al sistemului anterior dezvoltat  Profesor și Student, deși cu modificări minore continuă să-și îndeplinească rolul. Drept urmare, atât profesorul, cât și elevul au posibilitatea de a  urmări prezența la curs, iar profesorul are posibilitatea de a salva informațiile. 

Aplicațiile joacă un rol important în facilitarea contactului profesor-elev, și pe lângă faptul că  poate deține evidența prezenței, râmăn valabile și celelalte opțiuni cum ar fi trimiterea unui e-mail, crearea unui eveniment și posibilitatea de a intra în Google Classroom.

Pentru a rula modulul de identificare a elevului avem nevoie doar de un calculator, si totodata pentru a utiliza aplicația, aveți nevoie de telefoane cu sisteme de operare Android.
Baza de date utilizate este bazat pe un cloud, ceea ce reprezintă un acces mai ușor pentru permeabilitatea între module.


\vspace*{2cm}


\noindent \textbf{Cuvinte de cheie:} prezență, identificare, recunoaștere a feței, detectare a realității, aplicație Android

\vfill
\selectlanguage{english}
%\englishParagraph

%----------------------------------------------------------------------------
% Abstract in English
%----------------------------------------------------------------------------
\chapter*{Abstract}

Present-day educational system is far behind the modern era in some aspects. The majority of universities even today’s digital world still use traditional, paper based methods for managing student attendances. One of the biggest disadvantage of these methods is the lack of effectiveness. Furthermore, the later arrival students may disrupt the class while performing this procedure, not mentioning the absence of the records available for students afterwards.

The purpose of this project is to eliminate the previously implemented management software’s errors. Moreover, making it more reliable, faster and robust by replacing the identification module with a state-of-the-art neural network based facial image recognizer.

The system’s superiority is that the identification requires no special user interaction. Students only have to walk in front of a facial recognition camera. An appropriate facial recognition algorithm identifies the user, synchronously a liveness detection algorithm ensures the reliability of the system. In addition, users have the possibility to follow the attendance history.

The system consists of three main pillars. The attendance management module is responsible for identifying students and logging their data into the database. The existing Student and Teacher modules’ role remained the same, although contain minor changes. Appropriately both teachers and students have the opportunity to track attendances, while teachers can also export data.

The application takes an extensive role facilitating student-teacher contact. For this reason supplementary features added to the system, such as sending notification e-mails, creating calendar events and embedding Google Classroom access.

The facial recognition module requires a desktop environment, while the application runs on the Android operation system. The target database is cloud-based which contributes to an easier access and interoperability between the distinct modules.

\vspace*{2cm}

\noindent \textbf{Keywords:} attendance, identification, face recognition, liveness detection, Android application 

\vfill
\dolgozatnyelve
\defaultParagraph