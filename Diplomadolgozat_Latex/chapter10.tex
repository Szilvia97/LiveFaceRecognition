%----------------------------------------------------------------------------
\chapter{Továbbfejlesztési lehetőségek} \label{chapter10}
%----------------------------------------------------------------------------

Bár igyekeztünk a rendszert a lehető legjobb formában kivitelezni, figyelembe véve a korábbi észrevételeket és javaslatokat, mégis vannak olyan hiányosságok, amelyeket a jövőben szeretnénk pótolni.

Ezek közül megemlíthetjük a következőket:

\begin{itemize}
    \item Mérések és kimutatások készítése a rendszer megbízhatóságáról.
    \item Platformfüggetlen mobil alkalmazások: a rendszer részét képező mobilapplikációk Flutterben való átírása.
    \item Diákok képeinek mentése adatbázisba.
    \item Párhuzamosítás: az arcfelismerő és életszerűség-érzékelő algoritmusok párhuzamos futásának megvalósítása.
    \item A jelenlétkezelő modul szinkronizálása a hivatalos órarenddel.
    \item Admin felület: az adatbázist kezelő adminisztrátor szerepkörű felhasználónak egy felhasználói felület létrehozása a feladatok megkönnyítése végett.
    
\end{itemize}

A következő tanévben szeretnénk tesztüzembe helyezni a jelenlétkezelő rendszert, majd egy felmérést végezni a felhasználók körében. A felmérés célja, hogy visszacsatolást kapjuk a rendszer használhatóságáról, esetleges hibáiról. Mindemellett pedig nyitottak vagyunk minden olyan javaslatra, amely segít, hogy az általunk megvalósított élő arcfelismerés-alapú jelenlétkezelő alkalmazás a lehető legideálisabb formát öltse.