%----------------------------------------------------------------------------
\chapter{Bevezető}%\addcontentsline{toc}{chapter}{Bevezető}
%----------------------------------------------------------------------------

Egy 21. századi jelenség, hogy az embert digitális eszközök veszik körül az élet minden területén, ami a rohamosan fejlődő technikának köszönhető. Nincs ez másképp az oktatás területén sem. Nagyon fontos, bár nehéz vállalás, hogy a tanárok és a tanügyi rendszer is állja a sarat a modern technológiai kihívásokkal szemben.

Az új technológiai kihívásokkal szemben legkönnyebben a felsőoktatási intézmények, az egyetemek tudják felvenni a versenyt. Ez annak köszönhető, hogy az egyetemek és hallgatóik talán a lehető legegyszerűbben és leggyorsabban tudnak igazodni a változásokhoz, és a tapasztalatok azt mutatják, hogy törekednek is a fejlődésre, korszerű és új módszerek bevezetésével.

Ennek ellenére, még mindig sok az olyan tanintézmény, ahol a hagyományos jelenlétkezelő módszereket alkalmazzák az órák során. Ide tartozik a papírra való névsorírás, amikor a diákok maguk írják fel neveiket, de hasonló módszer az is amikor névsorolvasás történik, és a tanár végzi el a jelenlétkezelés adminisztratív részét. 
Egyöntetűen állíthatjuk, hogy ezen módszerek egyike sem hatékony és megbízható, mivel a diákok részéről könnyen átjátszható, ennek kiküszöbölésére pedig még több intézkedésre van szükség, ami növeli az erre szánt időkeretet. Emellett pedig az utánkövetés lehetősége egészen minimális, valamint többletmunkával jár a tanár részéről ezen adatok karbantartása. Mindegyik módszerre igaz az, hogy időtakarékosság szempontjából a lehető legrosszabbul teljesítenek. Az oktatásra szánt időből veszítünk el ezáltal hasznos perceket, amit a fejlődésre kellene fordítani. Fontos, hogy minden téren törekedjünk az ilyen és ehhez hasonló adminisztratív feladatok digitalizálására.

A fő kérdés, hogy mit kell szem előtt tartanunk egy jelenlétkezelő alkalmazás kivitelezése során? A válasz egyszerű: legyen hatékony, megbízható, könnyen kezelhető és mindenki számára hozzáférhető.

Kivitelezés szempontjából már nehezebb dolgunk van, hiszen fontos, hogy olyan rendszert alakítsunk ki, ami olyan szoftver és hardver eszközöket igényel, melyek könnyen elérhetőek és anyagi vonzatuk sem jelentős. Erre a legjobb megoldás az okostelefon, mint a rendszer részét képező eszköz bevonása, illetve a hiteles adatok tekintetében a biometrikus azonosítás alkalmazása. Emellett ami szintén nagyon fontos, hogy a rendszer használata érdekében a hallgató nem lehet kényszerítve arra, hogy rendelkezzen az eszközzel, illetve hogy a tanórák során használja azt. 

Számos irodalmi tanulmányban beszélnek arról, hogy hogyan és milyen számítástechnikai eszközöket vesznek igénybe az oktatás lebonyolítása során. Nagyon sok próbálkozás született már digitalizált jelenlétkezelő alkalmazás megvalósítására, amelyek más-más eszközt használnak a hatékonyság, biztonság és gyorsaság érdekében. Ilyen eszköz például RFID, ami bár hatékony, viszont kijátszhatósága megkérdőjelezhető, illetve ennek alkalmazása egy plusz eszközt igényel, amelyet minden diák számára biztosítani kell. 

A dolgozat során az ehhez hasonló problémákra keresünk megoldást, valamint ennek keretein belül szeretnénk megvalósítani azt a jelenlétkezelő rendszert, amely megfelel a technológiai elvárásoknak. 

Mivel sok esetben fontos az órai jelenlétek során egy meghatározott mennyiség megléte, nagyon fontos az is, hogy a hibalehetőségeket a minimálisra csökkentsük, ugyanakkor a megbízhatóság szempontjából a lehető legjobb megoldásra törekedjünk. Figyelembe kell vennünk azt is, hogy amennyiben lehet minimalizáljuk a beavatkozások szükségességét úgy a tanár mint a diák részéről. 

A modern jelenlétkezelő alkalmazás elvárt funkcionalitása a jelenlétek utánkövetésének lehetőség tanár és diák oldalon egyaránt. Ez egy olyan előnyt jelent a digitalizált megoldás részéről, amit egyik szokványos megközelítés sem tud biztosítani. Ezenkívül számos olyan előnnyel járhat, ami megkönnyíti, felgyorsítja és hatékonyabbá teszi az egyetemi órákon való jelenlétek menedzselését.

Az általunk fejlesztett élő arcfelismerés-alapú jelenlétkezelő alkalmazással szeretnénk hozzájárulni az egyetemen történő jelenlétkezelés új szintre való emeléséhez, eleget téve a felhasználók igényeinek és a technológiai elvárásoknak.
